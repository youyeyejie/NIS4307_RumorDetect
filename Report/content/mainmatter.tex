\chapter{任务目标}

本次课程设计的任务是基于谣言检测数据集,构建一个检测模型。
该模型可以对数据集中的推文进行谣言检测与识别。要求如下:
\begin{itemize}
    \item 数据集:使用给定的谣言检测数据集,数据集包含推文文本和标签(谣言或非谣言)。
    \item 训练模型:使用逻辑回归或GRU等深度学习模型进行谣言检测,实现二分类任务,
    用0代表非谣言、1代表谣言
    \item 泛化能力:模型应具有较好的泛化能力,能够适应不同类型的谣言检测任务。
    \item 评估指标:分类准确率、运行时间等
    \item 结果可视化:对模型训练结果进行可视化展示。
\end{itemize}

\vspace{1em}

我们需要在接口类文件classify.py中实现接口类RumourDetectClass,
该类对外提供一个接口函数classify,该函数接收一条字符串作为输入,
输出一个int值作为对应的预测类别。该类共包含以下方法:
\begin{itemize}
    \item \texttt{\_\_init\_\_(self)}: 初始化类
\end{itemize}

\chapter{具体内容}

\section{实施方案}


\section{核心代码分析}

\begin{codeblock}[language=Python]
import re
from train_gru import *

class RumourDetectClass:
    def __init__(self):
        # 加载词表和模型参数
        self.vocab = build_vocab(pd.read_csv('../dataset/split/train.csv')['text'])
        self.model = BiGRU(len(self.vocab), EMBEDDING_DIM, HIDDEN_DIM).to(DEVICE)
        self.model.load_state_dict(torch.load('../Output/bigru.pt', map_location=DEVICE))
        self.model.eval()

    def preprocess(self, text):
        # 文本预处理(与训练时一致)
        text = re.sub(r'[^\w\s]', '', text.lower())
        return text
    
    def classify(self, text: str) -> int:
        # 预测流程
        text = self.preprocess(text)
        ids = encode(text, self.vocab)
        x = torch.tensor([ids], dtype=torch.long).to(DEVICE)
        with torch.no_grad():
            logits = self.model(x)
            pred = (torch.sigmoid(logits) > 0.5).float().item()
        return int(pred)
\end{codeblock}


\section{测试结果分析}


\begin{figure}[ht]
  \centering
  \includegraphics[width=4cm]{example-image.pdf}
  \caption{示例图片}
  \label{fig:example}
\end{figure}




\chapter{工作总结}

\section{收获与心得}



\section{遇到问题及解决思路}

\chapter{课程建议}
