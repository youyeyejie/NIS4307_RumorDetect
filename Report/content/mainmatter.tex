\chapter{任务目标}

本次课程设计的任务是基于谣言检测数据集,构建一个检测模型。该模型可以对数据集中的推文进行谣言检测与识别。要求如下:
\begin{itemize}
    \item 数据集:使用给定的谣言检测数据集,数据集包含推文文本和标签(谣言或非谣言)。
    \item 训练模型:使用逻辑回归或GRU等深度学习模型进行谣言检测,实现二分类任务,
    用0代表非谣言、1代表谣言
    \item 泛化能力:模型应具有较好的泛化能力,能够适应不同类型的谣言检测任务。
    \item 评估指标:分类准确率、运行时间等
    \item 结果可视化:对模型训练结果进行可视化展示。
\end{itemize}

\vspace{1em}

我们需要在接口类文件classify.py中实现接口类RumourDetectClass,该类对外提供一个接口函数classify,该函数接收一条字符串作为输入,输出一个int值作为对应的预测类别。该类共包含以下方法:
\begin{itemize}
    \item \verb|__init__(self, model_path)|: 初始化,加载词表和模型参数
    \item \verb|preprocess(self, text)|: 对输入文本进行预处理
    \item \verb|classify(self, text: str) -> int|: 对输入文本进行预测,返回预测结果
\end{itemize}

\chapter{具体内容}

\section{实施方案}

在经过小组成员的讨论后,我们决定使用双向门控循环单元(BiGRU)模型开展谣言检测任务。该模型凭借神经网络的自动特征学习能力,无需人工设计文本特征,可直接从词语序列中提取语义特征,精准捕捉文本中的上下文依赖关系。相较于老师提供的另一种方法逻辑回归,BiGRU 模型具备以下显著优势:BiGRU模型可以同时捕捉文本的前向和后向上下文,进一步提升语义理解能力,尤其适用于分析谣言中常见的逻辑矛盾或因果关系表述;此外,BiGRU模型通过词嵌入层和循环神经网络结构,自动学习不同领域谣言数据的潜在特征分布,适应不同领域的谣言数据集,具有更好的泛化能力和鲁棒性。

\section{核心代码分析}

\subsection{训练模型train\_gru.py}
\begin{codeblock}[language=Python]
import re
from train_gru import *

class RumourDetectClass:
    def __init__(self):
        # 加载词表和模型参数
        self.vocab = build_vocab(pd.read_csv('../dataset/split/train.csv')['text'])
        self.model = BiGRU(len(self.vocab), EMBEDDING_DIM, HIDDEN_DIM).to(DEVICE)
        self.model.load_state_dict(torch.load('../Output/bigru.pt', map_location=DEVICE))
        self.model.eval()

    def preprocess(self, text):
        # 文本预处理(与训练时一致)
        text = re.sub(r'[^\w\s]', '', text.lower())
        return text
    
    def classify(self, text: str) -> int:
        # 预测流程
        text = self.preprocess(text)
        ids = encode(text, self.vocab)
        x = torch.tensor([ids], dtype=torch.long).to(DEVICE)
        with torch.no_grad():
            logits = self.model(x)
            pred = (torch.sigmoid(logits) > 0.5).float().item()
        return int(pred)
\end{codeblock}


\subsection{接口类classify.py}
\begin{codeblock}[language=Python]
import torch
import re
from train_gru import *

class RumourDetectClass:
    def __init__(self, model_path):
        # 加载词表和模型参数
        self.vocab = joblib.load(vocab_path)
        self.model = BiGRU(len(self.vocab), EMBEDDING_DIM, HIDDEN_DIM).to(DEVICE)
        self.model.load_state_dict(torch.load(model_path, map_location=DEVICE))
        self.model.eval()

    def preprocess(self, text):
        # 文本预处理(与训练时一致)
        text = tokenize(text)
        ids = encode(text, self.vocab)
        return ids
    
    def classify(self, text: str) -> int:
        # 预测流程
        ids = self.preprocess(text)
        x = torch.tensor([ids], dtype=torch.long).to(DEVICE)
        with torch.no_grad():
            logits = self.model(x)
            pred = (torch.sigmoid(logits) > 0.5).float().item()
        return int(pred)
\end{codeblock}

\section{测试结果分析}


\begin{figure}[ht]
  \centering
  \includegraphics[width=0.8\textwidth]{../Output/Graph/embedding_100_hidden_128_epoch_10.png}
  \caption{embedding\_100\_hidden\_128\_epoch\_10}
\end{figure}




\chapter{工作总结}

\section{收获与心得}

通过本次课程设计,我深入学习了深度学习模型在自然语言处理中的应用,特别是双向门控循环单元(BiGRU)模型在文本分类任务中的优势。通过对比不同模型的性能,我认识到模型的选择对任务结果的影响。此外,我还掌握了数据预处理、模型训练和评估等一系列技能,为今后的研究奠定了基础。

\section{遇到问题及解决思路}

在项目实施过程中,我遇到了一些问题,例如数据集不平衡导致模型偏向于某一类标签。为了解决这个问题,我尝试了数据增强和调整损失函数等方法,最终通过对训练数据进行重采样,取得了较好的效果。

此外,我还遇到了一些模型训练过程中的技术问题,例如梯度消失和过拟合等。为了解决这些问题,我尝试了不同的优化算法和正则化方法,最终通过调整学习率和使用Dropout等技术,成功提高了模型的性能。

\chapter{课程建议}

本次课程设计通过实践操作,让我们对深度学习与自然语言处理的结合应用有了具象认知,但在课程学习及实践过程中,也发现一些可以优化改进的方向,此处提出几点建议,希望能为后续课程设计提供参考:

当前课程较多聚焦于基础概念的介绍讲解,但对具体算法的原理推导与代码实现讲解较少,部分概念比较晦涩难懂但缺乏深入剖析,导致学生在理解上存在困难。希望老师能结合代码实例进行拆解演示,如结合 PyTorch 等库的具体实现,深入讲解 GRU、LSTM 等模型的工作原理与数学推导,帮助学生更好地理解模型背后的逻辑。

此外,本课程前期未铺垫相关实践案例,而课程设计在学期末才公布,与其他课程结课任务、考试复习等时间冲突,导致学生难以分配足够精力深入探索,也是我认为可以改进的地方。在本次课程设计中,我们小组成员普遍感受到时间紧迫,从理解任务、数据预处理到模型调优全程压缩在短时间内,尤其是在数据预处理、模型训练与调优等环节,难以进行充分的实验与探索。建议将课程设计主题提前半学期公布,分阶段设置任务节点(如第 8 周完成数据预处理、第 12 周提交模型初版等),并配套阶段性指导,帮助学生合理规划时间,确保实践质量。

而在完成项目的过程中,我们也意识到仅凭课堂上学到的知识,难以独立完成整个项目,特别是对 PyTorch 、Sklearn 等库的使用不够熟悉,导致在实现过程中遇到很多问题。建议老师能在前期的课程中结合具体案例,帮助学生更好地掌握这些工具的使用方法。